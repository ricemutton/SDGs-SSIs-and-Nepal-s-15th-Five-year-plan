\documentclass[a4paper]{article}
\usepackage{graphicx}
\usepackage{multirow}
\usepackage[
   top=10mm,
   bottom=20mm,
   left=20mm,
   right=20mm]{geometry}
\setlength{\parindent}{0pt}
\setlength{\parskip}{12pt}
\usepackage{titlesec}
\titleformat{\section}{\bfseries}{}{0pt}{\MakeUppercase}
\titleformat{\subsection}{\bfseries}{}{1em}{}
\titlespacing{\section}{0pt}{12pt}{-3pt}
\pagestyle{plain}
\title{Evaluating Policy Interventions for \\Small-Scale Industries in Achieving \\Sustainable Development Goals in Nepal}
\date{2023, May}
\author{Sharada Poudel - 40389\\ Department of Industrial Engineering\\ Institute of Engineering, Thapathali Campus\\
sharadapoudel2024@gmail.com}
\begin{document}
\maketitle
 
\section{Abstract}
The commitment to Sustainable Development Goals has directed Nepal’s policies to put a higher focus on small-scale industries. The paper includes qualitative and quantitative analyses of the impact of the policies. Qualitative analysis was carried out through KIIs and Quantitative analysis was carried out through secondary sources. Micro, Cottage, and Small-scale industries are collectively, addressed as small-scale industries (SSIs) throughout the paper. Upon study, it was found that through its policies, the government of Nepal is using SSIs as direct means to help in the achievement of SDG 1: No poverty, SDG 5: Gender Equality, and SDG 9: Sustainable Industrialization. Analysis of data shows that there was considerable growth in achieving the target of registration and women’s participation. However, industrial zones have not seen enough growth.

Keywords: Sustainable Development Goals, Small-scale industries (SSIs), Financial Inclusion

\section{Introduction}
Nepal is currently functioning in the 15th periodic plan (FY 2076/77-2080/81) as a part of planned development. It is the first periodic plan that integrates Sustainable Development Goals (SDGs) in Nepal’s national development framework. The commitment to achieve SDGs has led Nepal to focus on empowering its SSIs since their growth is related to the achievement of all 17 Sustainable Development Goals [1]. Small and medium-sized enterprises (SMEs) are the main vehicles for employment generation, poverty alleviation, and reduction of the economic and social inequalities of Least Developed Countries (LDCs) [2].

The study of policies related to small-scale industries is crucial because SSIs are highly affected by market failures and business environment constraints. Thus, they have a greater dependency on the policies governing them. Given their vital importance in maintaining inclusivity in the development phenomenon, providing them with a feasible environment to grow is indicative of a highly inclusive economy, especially for the LDCs where disparities are larger than in other countries.

The hierarchy of policies related to SSIs in Nepal is shown in the figure below. SDGs will be the guiding principle until 2030. The periodic plans of Nepal are guided by the country’s priorities. Nepal’s 15th plan has incorporated working policies that directly adds up toward the achievement of SDG goals. Industrial Enterprise Act 2076 is the acting policy in the present context. SDGs and periodic plans are the strategic frameworks and the execution done via Industrial Enterprise Act and various other policies.

The working policies in the 15th plan related to small-scale industries to achieve the related SDGs are mentioned in Table 1.

\begin{table}[h]
\centering
\begin{tabular}{p{3in}p{3in}}\hline
Working Policies & Related Sustainable Development Goals\\\hline
Micro, cottage, and small industries will be protected and promoted for employment creation and poverty alleviation in cooperation and collaboration with provincial and local levels. &
SDG-1 (No Poverty): End poverty in all its forms everywhere.\\\hline
Industrial estates and industrial villages will be established for women entrepreneurs to develop women’s entrepreneurship &
SDG-5 (Gender Equality): Achieve gender equality and empower all women and girls\\\hline
Priority will be given to soft loans for micro, cottage, small, and medium enterprises &
SDG-9 (Sustainable Industrialization): Build resilient infrastructure, promote sustainable industrialization, and foster innovation\\\hline
\end{tabular}
\caption{Working Policies of 15th plan and related SDGs.}
\end{table}

The governing act for the industrial sector, Industrial Enterprise Act 2076, has provisions and facilities for industries registered under this Act. The Act aims to facilitate industries by tax exemption in customs and income tax (50 \% of the income tax is to be exempted for fixed capital under 10 million for small-scale industries, and 100 for micro-industries), provision of different concessions, ease of registration, and redistribution of funds from industries of developed areas to underdeveloped areas, well-functioning industries to needy industries through different mediums. 

This paper focuses on the study of the working policies of the 15th plan and the growth of SSIs over the years under the current plan. It also relates the achievements with related SDGs. The selected indicators are the number of registration of industries per year, the number of women microentrepreneurs, the number of industrial areas, and the level of access to finance for small-scale industries. 

\section{Methodology}
The study used secondary data from the National Statistics Office and Department of Industry, and primary data sources too were utilized in gaining the actual scenario and include their recommendation in the study. Key Informant Interviews (KIIs) were conducted with officials from the Federation of Nepal Cottage and Small-scale Industries (FNCSI), Federation of Industries in Nepal Industrial Estate (FINIE), and Balaju Industrial Development Association. The questions used for the KIIs were:
\begin{enumerate}
\setlength{\labelwidth}{3em}
\setlength{\itemsep}{0pt}
  \item How well have Nepali small-scale industries performed in last 5 years?
  \item What is the problem being faced by SSIs in Nepal?
  \item How has the government tried to achieve SDG targets directly related to SSIs?
  \item How well have the existing policies in the country helped the industries?
\end{enumerate}


\section{Observation}
\subsection{Growth of the number of SSIs}
SDG1 (No poverty) focuses on eradicating extreme poverty by 2030. Micro and small business entrepreneurship contribute significantly. One of the major efforts from the government includes Micro Enterprise Development Programme (MEDEP). As the output, the growth in a number of industries during the last 5 years is shown in the graph below.

Source Yearly publication of “Micro, cottage, and small-scale industries statistics”

\begin{figure}
\centering
\includegraphics[width=8cm]{Plot}
\caption{Number of SSIs registered annually}
\end{figure} 

It is seen that the number of small industries registered in the year 2078/79 was 65069 making the total number of registered industries 5,87,802 till that year whereas the active registration was numbered in that year were 5,52,712. It shows that about 35,000 (5 percent) industries were not able to survive the market till 2078/79. This shows that registered industries have sustained themselves in the market in good proportion. 

\begin{figure}
\centering
\includegraphics[width=8cm]{Plot}
\caption{Number of SSIs with active registration}
\end{figure}
  

\subsection{Industrial Zones, 2080}
Industrial Zones are an efficient way to stimulate economic growth, by strengthening the competitiveness of enterprises [3]. These forms of clustering also help in identifying common sector-specific challenges and collectively identifying the strengths of SSIs [4]. Thus, the development of industrial zones is compared with the targets set in the 15th plan in the table below.
Table 2: Industrial Zones in Nepal
\begin{table}[!h]
\centering
\begin{tabular}{lll}\hline
Type & Updates 2021/22 & Target for 2023/24\\\hline
Industrial Villages & 57 Announced & 351\\\hline
\multirow{2}{*}{Special Economic Zone} & 1 Operational & \multirow{2}{*}{5} \\
& 2 Under construction &\\\hline
Industrial Estate & 10 operational & 14 \\\hline
\end{tabular}
\caption{Industrial Zones in Nepal}
\textit{ Source: Special Economic Zone, Nepal}
\end{table}


As shown by the data above, there is a huge gap in targets and achievements in case of industrial zones.
 
\subsection{The Scenario of Women Entrepreneurship}
SDG-5 with its focus on women’s empowerment looks after the level of participation of women in economic activities. Specifically, SDG Indicator 5.5.2, (Proportion of women in managerial positions) is a close indicator to the scenario of women entrepreneurship.  Governmental efforts such as capacity-building programs and Women Entrepreneurship funds are found to be effective in reducing the gender gap in ownership of industries. 53.48\% of the enterprises were registered with ownership of women in FY 2078/79. The report “National Economic Census 2018” by the National Statistics Office concluded that women-owned 23\% of the businesses including the informal sector. Around the same time, in the formal sector, the number of registration of women-owned businesses was 28\% of total registration. Inclusivity is seen more on formal sector. Data including the informal sector is yet to be released.
\begin{figure}
\centering
\includegraphics[width=8cm]{Plot}
\caption{Scenario of Women Entrepreneurship}
\end{figure}
Source: Micro, Cottage, and small-scale industries statistics
\subsection{Access to finance}
SDG-9, target 9.3 is focused on the issue of access to finance for small-scale industries. Access to finance is also one of the major working policies mentioned above. The government has been allocating amounts (Rs. 13.59 billion in the year 2080/81 and Rs. 13 billion in 2079/80) for interest subsidies on concessional loans for micro, cottage, and small-scale industries. The total interest subsidy provided by the government for concessional loans reached Rs. 22,390,516,090 in the year 2079 at the end of Chaitra. The subsidy doesn’t favors only SSIs but a fraction of it helps in easing the access to finance for SSIs. Nepal Rastra Bank has mandated commercial banks to allocate at least 11 percent of the total loan to the small, micro, cottage, and medium industries by mid-July 2022 which is to increase in coming years. As of mid-April 2022, Rs.393.28 billion (9.85\%) loan has been disbursed in the sector [5]. However, financial access is not distributed evenly. The bigger business houses have easy access to financial services than the smaller ones [6]. 
\section{Result}
Thus, it is seen that there has been an achievement of targets set in the 15th plan despite slack caused by the COVID-19 pandemic, and women’s participation too has increased significantly. The number of active registered industries increased by 2 million in 4 years till 2078/79. However, the government has been unsuccessful in the execution of targets set for industrial zones. It is also found that the plans set are not realistic, where each year the number of industrial zones increases. Building industrial zones takes time and different stages of different zones can be carried out simultaneously, but this pattern is not seen in the planning phase. With 53\% of registered industries of 2078/79 being owned by women, good growth was seen. Also, plans are seen to be executed well in case of increasing access to finance. 
\section{Recommendations}
The recommendations are presented based on KIIs conducted. The recommendations made during KIIs are:
\begin{enumerate}
\setlength{\labelwidth}{3em}
\setlength{\itemsep}{0pt}
  \item Study-based policymaking should be given high priority.
  \item The implementation of policies is the major point of concern, thus efforts on implementation should be increased.
  \item There should be effective means of communication between different levels of government.
  \item Small-scale industries should focus on promoting their product more, and government should create a feasible environment for that.
  \item The progress reports of these plans should be published to understand the level of effort required.
  \item The means of measurement too should be assessed on the basis of the impact they have on the economic level. Targets should be focused on value addition by small-scale industries.
\end{enumerate}
 
\section*{References}

[1] "MSMEs and their role in achieving the Sustainable Development Goals," United Nations Department of Economic and Social Affairs, 2020.

[2] B. Blum, V. LaFleur, R. C. Blum and S. Talbott, "Expanding Enterprise, Lifting the Poor: The Private Sector in the Fight against Global Poverty," Brookings Institution, 2005.

[3] N. Stanojević, "Importance of industrial clusters for Middle East economy," Industrija, pp. 77-96, 2008. 

[4] E. Jimenez, M. d. l. Cuesta-Gonzalez and M. Boronat-Navarro, "How Small and Medium-Sized Enterprises Can Uptake the SDGs through Cluster Management Organization: A case study," Sustainability, 2021. 

[5] "Monetary Policy," Nepal Rastra Bank, 2022/23.

[6] P. Kharel and K. Dahal, "Small and Medium-sized enterprises in Nepal: Examining Constraints on Exporting," 2020.

[7] N. A. Adebayo and M. L. Nassar, "Impact of Micro and Small Business Entrepreneurship on Poverty Reduction in Ibadan Metropolis, South Western Nigeria," International Review of Management and Business Research, 2014. 

\end{document}



